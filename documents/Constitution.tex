\documentclass{report}
\usepackage[cm]{fullpage}

% Watermark
\usepackage{draftwatermark}
\SetWatermarkText{PROPOSAL}

% Hyperlinks
\usepackage[usenames,dvipsnames]{xcolor}
\usepackage[pdftitle={The Aberystwyth Live Action Games Society Constitution}, pdfauthor={The AberLAG Committee}, pdfcreator={Sam Clements}]{hyperref}
\hypersetup{colorlinks=true, linktocpage=true, linktoc=all, linkcolor=Black, urlcolor=Black} 

% Set font
\renewcommand{\familydefault}{\sfdefault}

% Refrences
\newcommand{\sref}[1]{\hyperref[#1]{Section \ref*{#1}}}
\newcommand{\aref}[1]{\hyperref[#1]{Appendix \ref*{#1}}}

% Society name
\newcommand{\society}{Aberystwyth Live Action Games Society}
\begin{document}

% Title page
\title{The \society{} Constitution}
\author{Authored by the \society{} Committee}
\date{October 2012}
\maketitle{}

\tableofcontents
\newpage

% --------------------------------------------------
\chapter{The Society}
% --------------------------------------------------

\section{General}

The name of the Society shall be the \textbf{\society} (hereafter referred to as `{\em the society}') and shall be allowed to be abbreviated to \textbf{AberLAG} in writing to save both ink and paper. \\

\noindent All previous Constitutions are hereby null and void.

\section{Mission}

The mission for the \society{} shall be to encourage teamwork, critical thinking and camaraderie between its members via activities run by the society.

\section{Guild and BOS rules}

The society shall be up to date, and in compliance with all rules and regulations concerning Aberystwyth Guild of Students (hereafter referred to as the Guild), and the Aberystwyth Guild of Students Board of Societies (hereafter referred to as BOS). This includes both the Guild and BOS constitutions.

% --------------------------------------------------
\chapter{Members and Committee}
% --------------------------------------------------

\section{Membership}

The society shall have these forms of membership:
	
	\subsection{Committee members}
	\label{member_committee}
	
	The members of the committee are responsible for the running and organisation of the society.
	
	The committee must be made up of the positions described in Section \ref{committee}. Each of these positions is an elected position and the member must be elected as described in Section \ref{elections}.	
	
	Both elected and appointed committee members must quality for Ordinary Membership.
	
		\subsubsection{Appointed Committee members}
		
		The committee may also create and appoint additional positions, if they agree that doing this would benefit the society. These positions last until the next full election, at which point the new committee must decide whether to appoint them again, or until the elected committee decides to remove the appointed committee member from their position.
		
		At no point may the number of appointed positions outnumber the elected positions.
	
	\subsection{Ordinary members}
	\label{member_ordinary}
	
	Ordinary Membership of the society will be the normal level of membership available to anyone who is an Ordinary members of the BOS, as defined in \href{http://www.aberguild.co.uk/en/sports-a-activities/societies/bos-constitution}{Section 3 of the BOS constitution} and has paid current membership fees (\sref{fees}).
	
	Standard members will be allowed to participate in events and may stand for election (\sref{elections}).

	\subsection{Associate members}
	\label{member_associate}
	
	Associate membership of the society will be available to those who do not fill the requirements to become an Ordinary Member (including those not registered with the university). The committee shall have the right to deny any person associate membership.
	
	Associate Members hold the same rights as Ordinary Members but may not stand for election or be appointed to a committee position.
	
\newpage
\section{Committee}
\label{committee}

The Committee shall be made up of the following positions: President, Vice-President, Secretary, Treasurer, Events Manager, and Health and Safety Officer. All committee members are bound to carry out both the duties defined below, as well as any additional duties defined within the rest of the document.

	\subsection{President}

	The President shall be the main contact point for external communications with all external groups and individuals, including the Guild. They are further responsible for the smooth and effective running of the Committee and the society, and shall be the final arbiter of disputes amongst the Committee.
	
	They are responsible for creating the best environment for the Committee and other members to operate in and must ensure the general health and direction of the society is in accordance with the views of its members.
	
	\subsection{Vice-President}
	
	The Vice-President shall be the secondary contact point alongside the President.
	
	They are also responsible for standing in or appointing acting members of the Committee should the need arise.
	
	\subsection{Secretary}
	
	The Secretary is responsible for organising all meetings related to the running of the society, and are responsible for recording the minutes of all meetings.
	
	They are the main contact for receipt and outgoing administrative details of all events with regards to members and external organisations (including the Guild).
	
	\subsection{Treasurer}
	
	The Treasurer is responsible for all of the societies finances.
	
	This includes collecting fees for both membership and society events requiring a fee. They shall be responsible for keeping records of all financial transactions throughout the academic year. Any outgoing funds require the approval of the Treasurer.
	
		\subsubsection{Accounts}
		
		The finances of the society shall be held primarily by the Treasurer, and secondly by the Events Manager. The Treasurer shall manage the accounts in an annual fashion with the accounting year ending in accordance with the handover of responsibilities each year and shall keep and maintain all records of the society accounts before passing them on to their successor.
		
		The society shall maintain all of its funds within the Guild Finance Office. All monies received from members and other sources shall be paid into the society account within the Finance Office in accordance with the Guild rules and within one day of the finance office being open for business after the monies are collected. All outgoing funds must have the approval of the Treasurer and whomever else is on the Guild Finance mandate form.
		
		All financial actions undertaken by the society must be in accordance with the Guild and BOS financial rules, regulations and Constitutions.
	
	\subsection{Events Manager}
	
	The Events Manager is responsible for the smooth-running of society events.
	
	In particular, they are there to ensure that the all events are overseen and organised to their satisfaction, for both regular and major events. They should act as a direct contact point for feedback during events, which can then be passed on to the committee and event organisers.
	
	\subsection{Health and Safety Officer}
	
	The Health and Safety Officer is in charge of making sure all society events comply with the Guild’s Health and Safety guidelines, as well as making sure all related emergencies are dealt with safely and as promptly as possible.
	
	The Health and Safety Officer is also responsible for ensuring all equipment used in the societies activities is safe for use. On this matter only, they are the final arbiter, and all decisions on this subject are required to go though them.

\newpage
\section{Elections}
\label{elections}

	All members (excluding Associate members) of the society have the right to vote the societies elections. Those members which can vote are hereafter refereed to as {\em voting members}. Elections shall take place in accordance to this constitution, following the rules defined below.
	
	The committee must choose a returning officer to run all elections. They shall be responsible for totalling the votes, and announcing the results. The winner(s) of the election shall be decided by a simple majority vote.
	
	All elections shall take place no less than 14 days after the notification has been sent out to all members of the date, time and place.
	
	\subsection{Vacancies}
	
	Upon any vacancy occurring outside of normal election procedure, the Vice-President must decide on another member of the Committee to temporarily hold the vacant post until a by-election is completed for the vacancy.
	
	\subsection{Annual elections}
	
	An election shall be held at the end of each academic year for the entire set of committee positions. The returning officer for the Annual Election may not be one of the candidates.
	
	\subsection{`Postal' votes}
	
	For all elections, members with the right to vote in society elections (hereafter described as `voting members') must be provided with an alternate way to vote. For a period starting no less than one week before the election, up to the start of the election, the Returning Officer must accept letters from voting members describing their vote. A list of members who have voted by this method should be kept, so that they may not vote a second time during the election. This enables those members who may be unable to attend the elections to still have a vote.

\subsection{Vote of no confidence}

A vote of no confidence is a vote between members for the immediate dismissal of an elected or appointed committee member.

All voting members of the society have the right to propose a vote of no confidence, at any point, if either the majority of the Committee agrees, or a petition of signed by 10 voting members or a majority of the voting members (whichever is lower) agrees. The vote of no confidence must then be held at the following meeting - all members of the society must be informed at least one week before this, including the committee member who is being voted upon. The vote of no confidence shall be decided with a simple majority vote.

A vote of no confidence may not be called until a period of no shorter than one month of term time has passed since the election the committee member was voted in at.

\newpage
\section{Meetings}
\label{meetings}

	\subsection{Committee Meetings}
	
	A general meeting should be held at least once a month, if not more, in order to keep the amount that must be discussed at the meeting short. This should be an opportunity for all members to bring up issues or points they may have.

	\subsection{General Meeting}
	\label{agm}

	An General Meeting shall be held at the end of the academic year. The agenda of the General Meeting shall be:

	\begin{enumerate}
		\item To consider any motions that have been brought forward that require consideration before any further points are made.

		\item To receive reports from all the committee members on their activities in the last year.
		\item To receive a financial report, including the accounts of the previous financial year and a budget for the current financial year.
		\item To formally handover the elected positions from the outgoing committee to the incoming committee.
		\item For any constitutional amendments to be ratified, and the constitution to be approved as accurate for the forthcoming academic year.
		
		\item To consider any motion brought forward during the meeting or added to the set agenda.
	\end{enumerate}
	
	The Secretary should give at least 14 days notice of the time, data and place of the General Meeting.
	
	The General Meeting should follow all rules that apply to regular meetings, though those declared above take priority.	

	\subsection{Emergency General Meeting}
	
	The Emergency General Meeting will be an emergency procedure used when the survival of the society comes into question. The committee shall call an EGM when in their discretion a `serious emergency' arises. Once called, information regarding the meeting must immediately be announced to all members at least 14 days in advance. During the meeting, all ordinary members and above will have an equal vote on the future direction of the society.
	
% --------------------------------------------------
\chapter{Activities}
% --------------------------------------------------

The society will run multiple types of activities. These include both ``shootouts'' which require no special organisation, and ``scenarios'' which are larger events that may require special organisation and rules. The society may also run activities that do not fit the below descriptions.

\section{``Shootouts''}

Shootouts are minor events, that should require no other organisation than a time and date, and moderators only if the event will have a large number of people attending.

All shootouts must follow the rule-book, which shall be made available in a separate document.
This document - `The Aberystwyth Live Action Games Society Rulebook' - shall include both the Basic Rules and the Scenario Rules (see below).

	\subsection*{Moderators}
	
	Moderators shall be appointed by the elected committee only, and must be either a committee member or an ordinary member.
	
	The moderators will have additional powers during ``shootouts'' and will be able to oversee and enforce rules, with the power to impose minor penalties. They are expected to have knowledge of the societies rules, and may be asked to moderate for events that require a fixed set of moderators.

\section{``Scenarios''}

Scenarios are major events, that may require special organisation and have additional rules specific to the scenario. In addition to the Basic Rules, scenarios will have an additional set of rules - the Scenario Rules - that they must follow.

	\subsection*{Moderators}
	
	Scenarios will have a defined group of moderators for the entire event, selected by the committee. These moderators are not necessarily the same ad the moderators appointed for general ``shootout'' events. These moderators must not take part in the event.

% --------------------------------------------------
\chapter{Constitution}
% --------------------------------------------------

\section{Documentation}

A copy of this constitution shall be kept by the Secretary and distributed to members upon request. The Students Activities Officer shall also hold a copy, which must be submitted annually in a paper format. A copy shall also be put on the website.

\section{Amendments}

Amendments to this constitution happen in the following fashion. The proposed Amendment must be submitted to the Committee not within 14 days of a General Meeting. The Committee will then discuss the proposed amendment and then suggest any changes that would be suitable. The amendment is then proposed at the General Meeting where it must be ratified by a majority. Changes to policy (see \sref{policy}) are not subject to these rules, and may be changed as soon as they are ratified at a committee meeting.

Any changes to this constitution shall only come into force after the close of the meeting unless otherwise stated in the amendment or constitution.

\section{Interpretation}

In the event of any question of interpretation arising the Committee shall have the power to act accordingly to its interpretation of the constitution, or, if it does not cover the issue, then a constitutional amendment should be presented by the Chair at the next Committee Meeting.

	\subsection{Language}
	
	This constitution should be made available in both Welsh and English.
	
	Should there be any dispute over the constitution between the Welsh version and the English version, the English version shall take precedence.

\section{Approval}

The Constitution must be approved by a majority in order to stand as a true and accurate authority of the members of the society for the forthcoming academic year.

This constitution shall come into force at the start of the 2012-2013 academic year, to allow for amendments to be made during the summer period.

\appendix
\chapter{Policies}
\label{policy}

\section{Age of Associate Members}

While non-university students are welcome to join as Associate Members, we will require that all associate members are at least 18 years old\footnote{This is due to potential legal issues that we would not be able to deal with, as well as the general issues involved with having younger players.}.

\section{Unoffical Shootouts}

Shootouts or similar events organised by other players and groups are encouraged, but we require that anyone wishing to advertise through us must comply with (at minimum) our safety rules, in cases where they apply.

Shootouts that do use our rules, and that are organised by the societies members have permission to use the 'unofficial' \society{} logo (a grey-scale version of the official logo) in their advertising.

\section{Online Communication}

Online communications (such as the societies Facebook page) shall be subject to a ``three strikes'' rule, in order to keep communications friendly.
The Committee shall mark a ``strike'' against people who make inappropriate or rude comments, and any member reaching three ``strikes'' shall be given a temporary ban from the online community.
The length of the ban will be a minimum of 3 days, and will double for each previous ban the user has been given.
The Secretary shall be in charge of administrating these procedures.

\end{document}
