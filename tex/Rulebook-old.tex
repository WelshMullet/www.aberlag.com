\documentclass{report}
\usepackage[cm]{fullpage}

% Watermark
\usepackage{draftwatermark}
\SetWatermarkText{PROPOSAL}

% Hyperlinks
\usepackage[usenames,dvipsnames]{xcolor}
\usepackage[pdftitle={The Aberystwyth Live Action Games Society Rulebook}, pdfauthor={The AberLAG Committee}, pdfcreator={Sam Clements}]{hyperref}
\hypersetup{colorlinks=true, linktocpage=true, linktoc=all, linkcolor=Black, urlcolor=Black} 

% Set font
\renewcommand{\familydefault}{\sfdefault}

\newcommand{\sref}[1]{\hyperref[#1]{Section \ref*{#1}}}
\newcommand{\aref}[1]{\hyperref[#1]{Appendix \ref*{#1}}}

\newcommand{\society}{Aberystwyth Live Action Games Society}

\newcommand{\itemtitle}[1]{\textbf{#1} \hfill \\}

\begin{document}

% Title page
\title{The Aberystwyth Live Action Games Society Rulebook}
\author{The AberLAG committee \\ This document is a proposal.}
\date{May 2012}
\maketitle{}

\tableofcontents
\newpage

% --------------------------------------------------
\chapter{Player Rules}
% --------------------------------------------------

\section{Shootouts}

The shootout rules apply to all shootouts and scenarios.

	\subsection{Safety}
	
	The safety rules come above all others - breaking any of them result in a temporary or permanent ban from society activities\footnote{I.e. shootouts and scenarios, but not meetings.}.

	\begin{description}
		\item[Moderators] Listen to the moderators, and follow their instructions - especially at the start of scenarios and missions.
		\item[Dangerous areas] Tags made in a dangerous area (such as a busy road) are immediately invalidated and should always be avoided. Vehicles also count as a dangerous area, and no tags can be made to or from a vehicle (i.e. You can not shoot from a vehicle, or shoot at a player in a vehicle).
		\item[Physical contact] Don't get into physical contact - don't push, shove or grab at other players. The exceptions to this are LARP (or similar) melee weapons if they have been explicitly allowed, as described below.
		\item[Act sensibly] It may not be covered by the rules. But if could cause danger to another player, or get the game banned: \\ \textit{Don't do it.}
	\end{description}
	
	\subsection{General Rules}
	\begin{description}
		\item[Start and end times] Games start and end with the blowing of an whistle [{\em sic}], or a call of {\em START} and {\em STOP} from a moderator.
		\item[Act with honour, integrity and sportsmanship] Don't cheat, ignore the rules and/or moderators, or do anything else which will (or is likely to) spoil the game for other players.
		\item[Inactive or `dead' players] If you are out (i.e. no lives left, inactive, tagged etc), you may not interfere with the people who are still in the game. To signify that you are inactive, hold up either two fingers or your weapon in the air. 
		\item[Clean up] At the end of games, help collect dropped darts, and once the darts have been collected return them to their owners. Marking your name or a unique symbol on darts you want to make sure you keep is advised.
    \end{description}

    \subsection{Tagging}

    A tags is defined as a projectile hitting a player and coming to a stop/bouncing off, or a melee weapon touching another player. Tags that hit a weapon (and not the player) do not count, as do tags that hit feet. When tagging another player, you should call 'HIT' to clearly inform the tagged player that they have been tagged.
    
	Players can be tagged with a selection of weapons:

	\begin{itemize}
		\item \itemtitle{Dart weapons} Any foam dart/disc weapons may be used to tag other players. Modded weapons should be checked by moderators to ensure they do not hurt players.
		\item \itemtitle{Thrown weapons} Balled up socks (or similar small, soft items) can be used as thrown weapons.
    	\item \itemtitle{Melee weapons} Melee weapons (such as LARP safe swords) are restricted to contained events only (moderated events in a single area), and only if the moderators have allowed them. As with modded weapons, any users of a melee weapon must be safed by the Health and Safety officer.
	\end{itemize}
	
	\noindent A full list of legal weapons is available from the committee - if a weapon is not on this list, it is not a legal weapon.

\section{Scenarios}

The event rules apply to all society scenarios (e.g. large games such as HvZ). Only scenarios must comply with them.

	\subsection{Additions to the shootout rules}
	
	\begin{description}
		\item[Inactive or `dead' players] \hfill \\ If you are out, you may not interfere with the people who are still in the game. This includes not communicating information relevant to the game (especially information gained from speculating while inactive) to the active players, with the exception of communicating a) that you have died, and b) the location of your death.
		
		\item[Melee weapons]  \hfill \\ Melee weapons may not be used for scenarios, unless the committee has allowed them for the scenario or part of the scenario.
    \end{description}
    
    \subsection{Identification}
	
	Most scenarios will use bandannas or armbands to mark players that are in the game, and their team. For scenarios that does this, you must wear it at times in a visible location, unless you are dead or otherwise inactive. You may not tag any player who is not wearing such a marker under any circumstances\footnote{While this can be used by players to `cheat' and avoid being tagged, it is required so that players can remove themselves from the game in emergencies. If a player does abuse this, it should be reported to the committee and dealt with as a rules violation on their part - instead of on your part for tagging a person not marked as taking part in the game.}.
	
	\subsection{Camping}

	Camping is defined as ``{\em blocking an entrance to a building (or section of a building), such that a player cannot enter or exit the building without an immediate risk of being tagged''}. Players who are camping may be shot at from inside the building (or section), as well as tags made by campers having the possibility of being invalidated or otherwise punished.
	
	\subsection{Safe areas (outside games)}

	Tags made in off-limits areas are invalid. Off-limits areas include:
	
	\begin{itemize}
		\item All indoor areas, especially public areas such as cafeterias and shops.
		
		\item Places of worship and workplaces.
		\item Roads, or other areas likely to cause danger to players.
	\end{itemize}
	
	\subsection{Safe areas (indoor games)}

	Some games may allow indoor combat, as well as outdoor combat (generally games without open warfare). For these games, off-limits areas include:

	\begin{itemize}
		\item Lecture theatres when a lecture is in progress or within 10 minutes of a lecture.
		\item Restaurants, canteens and cafes, as well as any shops.
		\item Libraries and study areas.
		\item Bathrooms and changing rooms.
		\item Personal rooms or offices are off limits - unless you have been invited in by the owner\footnote{Often known as the Vampire rule}.
		\item Places of worship and workplaces.
		\item Roads, or other areas likely to cause danger to players.
    \end{itemize}

% --------------------------------------------------
\chapter{Moderator Rules}
% --------------------------------------------------

These rules are only needed for members who are moderating activities - players are not required to know them.

\section{Shootouts}
	
	\subsection{Moderation}

	Shootouts should have 1 moderator for each full ten players at the activity. The position of moderator is defined in the constitution, in Section 3.1 - ``Shootouts''. During a shootout, moderators can:

    \begin{itemize}
		\item Call PAUSE and PLAY to start and stop gameplay (e.g. if an accident happens, or a bystander needs to travel through the playing area).
		\item Start and stop a match, either by the blowing of an whistle, or by calling START and STOP.
	\end{itemize}
	
	\noindent The moderators should also:
		
	\begin{itemize}
		\item Organise the matches played, and the teams for them.
		\item Ensure that all darts are cleared up at the end of matches, and borrowed equipment returned to it's owner.
		\item Ensure that no unsafe equipment is being used.
	\end{itemize}

\section{Scenarios}

	Event moderators have the same powers as 'normal' moderators. However, the position is specific to the event, in the same way as the Head and Vice Head Moderator positions are (see below). They should have full knowledge of the event rules, and be willing to help out during missions and other periods of heavy activity during events.

	\subsection{Moderation}

	Scenarios should have both a Head Moderator and a Vice Head Moderator, chosen by the Committee. They take responsibility for the event for it's duration, and have the final say on disputes over rules.
	
	Scenarios should have additional Moderators if needed. Unlike the Head and Vice Head Moderators, they are not restricted from being able to player.

	In the case of both the Head and Vice Head Moderators not being able to attend all or part of an event, they should decide upon another person to take the role during their absence (for example, if the Head Moderator cannot attend a mission during the event, they should decide upon another person to take their role during the mission).

	\subsection{Indoor and outdoor games}
	
	Scenarios should generally run ``outside'' (in the context of rules for off-limits areas). Any scenarios that wish to run ``indoors'' must have the full permission of the committee.

\end{document}